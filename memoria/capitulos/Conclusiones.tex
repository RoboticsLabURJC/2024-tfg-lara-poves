\chapter{Conclusiones}
\label{cap:capitulo5}

El objetivo principal de este \ac{TFG} era lograr los tres tipos de comportamientos autónomos clave con algoritmo de \ac{DRL}. Este proceso ha sido una tarea complicada, sobre todo a la hora de encontrar los hiperparámetros de entrenamiento y diseñar funciones de recompensa que proporcionasen resultados eficientes. No obstante, también se ha evidenciado el potencial que tienen este tipo de algoritmos en tareas de conducción autónoma, optimizando la toma de decisiones en tiempo real y demostrando una gran capacidad de generalización en diferentes escenarios.

En esta sección se analizan los objetivos cumplidos y las competencias obtenidas durante el desarrollo de este \ac{TFG}, además de posibles líneas futuras del proyecto.
\section{Objetivos cumplidos}

Los objetivos presentados en la sección \ref{sec:descripcion} se han alcanzado con éxito.

\begin{enumerate}
\item Se ha logrado un comportamiento sigue-carril eficiente basado diferentes algoritmos de \ac{DRL}, tanto en \ac{DQN} como en \ac{PPO}, con ambas técnicas de detección de carril.
\item Se ha conseguido un modelo con \ac{PPO} capaz de adaptarse a la velocidad del coche delantero basándose en las mediciones de la zona frontal del \ac{LiDAR}, manteniendo el seguimiento efectivo del carril.
\item Se ha diseñado un modelo basado en \ac{PPO} capaz de realizar maniobras completas de adelantamiento, utilizando la información proporcionada por todos los elementos de percepción: detección de carril, diferentes subzonas del \ac{LiDAR} y segmentación de la calzada.
\item Se ha llevado a cabo un análisis y comparación de cada uno de los comportamientos de conducción autónomos obtenidos.
\end{enumerate}

\section{Requisitos satisfechos}

En la sección \ref{sec:requisitos} se propusieron los requisitos de este \ac{TFG}, los cuales se han solventado de la siguiente manera:

\begin{enumerate}
\item Se ha utilizado el entorno de simulación CARLA para el desarrollo de los diferentes comportamientos autónomos.
\item Se ha obtenido un modelo capaz de seguir el carril de forma fluida alcanzado velocidades de 20m/s.
\item El modelo de control adaptativo al tráfico además de ajustar su velocidad a la del vehículo delantero, lograr también regular automáticamente la distancia de seguridad, aumentando la separación si la velocidad del vehículo aumenta.
\end{enumerate}

\section{Balance global y competencias adquiridas}

Durante el desarrollo de este \ac{TFG}, el reto de implementar una aplicación de navegación autónoma en entornos urbanos capaz de ofrecer diferentes habilidades de conducción, ha sido un proceso complejo, pero enriquecedor. Combinar algoritmos de \ac{DRL} para la toma de decisiones con técnicas avanzadas de \ac{IA} para la percepción, ha resultado ser una solución prometedora que, con mayor estudio y progreso, puede llegar a ofrecer modelos robustos y eficientes en entornos reales y variados de conducción.

Al inicio de este proyecto, apenas tenía conocimientos básicos sobre \ac{DRL} y el alcance que podían tener soluciones basadas en redes neuronales para la percepción del entorno. Este ha sido mi primer gran trabajo en el campo, durante su desarrollo he aprendido numerosos conceptos y técnicas de desarrollo y análisis de resultados.

Destacando las siguientes competencias:

\begin{itemize}
  \item Mejorar la organización y planificación gracias a la metodología \textit{scrum}.
  \item Nuevos conocimientos en el uso de simuladores robóticos.
  \item Nueva concepción en el ámbito de la percepción, por ejemplo, la diferencia entre clasificación, detección y segmentación en imágenes.
  \item Profundizar en la construcción, entrenamiento y tipos de redes neuronales.
  \item Afianzar mis conocimientos sobre aprendizaje por refuerzo y ampliarlos con el aprendizaje de algoritmos \ac{DRL}.
  \item Capacidad para analizar resultados y gráficas basadas en los datos recolectados durante los entrenamientos y pruebas, con el fin de optimizar y mejorar los modelos autónomos desarrollados, así como detectar y abordar los problemas surgidos durante el proceso.
  \item Generación de documentación detallada, mejorando la presentación y explicación de los procesos y resultados obtenidos gracias al desarrollo de la memoria.
\end{itemize}


\section{Líneas futuras}

A pesar de que se han obtenido resultados eficientes y satisfactorios en el desarrollo de este \ac{TFG}, se proponen líneas futuras de crecimiento a partir de este trabajo. 
\begin{itemize}
    \item Probar otros algoritmos de \ac{DRL} que utilicen un espacio de acciones continuo, con el fin de comparar los resultados y evaluar si se obtienen mejores comportamientos.
    \item Hacer más segura la maniobra de adelantamiento, además de comprobar la existencia de carril a la izquierda, verificar si se puede llevar a cabo el cambio de carril de forma segura.
    \item Mejorar el adelantamiento (dependiendo un poco de lo que me dé tiempo a hacer).
    \item Adaptar los diferentes comportamientos autónomos a escenarios menos controlados y condiciones adversas, climas menos favorables, circuitos más complejos o situaciones de tráfico más elaboradas.
	\item Incorporar sensores como el radar y fusionar los datos de todos ellos para las diferentes habilidades de percepción, con el objetivo de obtener comportamientos más robustos del sistema autónomo. Si un sensor falla o proporcionan mediciones con mucho ruido, se podrá contrastar la información entre los sensores para detectar y corregir posibles errores, mejorando la fiabilidad del sistema.
\end{itemize}
