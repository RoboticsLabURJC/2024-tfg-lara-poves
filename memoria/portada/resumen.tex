\cleardoublepage

\chapter*{Resumen\markboth{Resumen}{Resumen}}

La conducción autónoma se ha consolidado como una de las áreas más destacadas dentro de la robótica, estrechamente vinculada a los avances significativos en inteligencia artificial logrados en los últimos años. La inteligencia artificial, junto con el aprendizaje automático y el aprendizaje profundo, impulsa el desarrollo de tecnologías innovadoras que se basan en información proporcionada por sensores y cámaras para navegar de forma precisa. 

Estas técnicas permiten la implementación de soluciones avanzadas que facilitan la percepción e interacción con el entorno, tomando decisiones informadas en tiempo real. Los modelos predictivos generados buscan optimizar la seguridad, la eficiencia y la sostenibilidad en el sector del transporte, aspectos fundamentales en la evolución de los sistemas de movilidad autónoma. Por ejemplo, estos avances tienen aplicaciones en la gestión del tráfico urbano o la automatización de flotas de transporte público.

Este TFG se centra en desarrollar un sistema de conducción autónoma basado en IA y aprendizaje por refuerzo profundo (\textit{Deep Reinforcement Learning}), cuyo objetivo principal es lograr comportamientos específicos únicamente mediante la definición de recompensas y penalizaciones. El propósito final de este trabajo es implementar con éxito una maniobra de adelantamiento, logrando previamente comportamientos básicos como el seguimiento de carriles y la conducción segura detrás de otro vehículo sin colisión.

Para la detección del entorno, se utilizan técnicas avanzadas de IA, como redes neuronales de segmentación semántica y redes neuronales específicas para la detección de las líneas de los carriles. Estas técnicas se combinan con algoritmos de aprendizaje por refuerzo profundo, como DQN y PPO, para que el vehículo aprenda a tomar las mejores acciones de control en función de la información sensorial obtenida en cada instante.

Todo este trabajo se ha desarrollado en el simulador CARLA, una plataforma ampliamente utilizada para investigar y probar sistemas de conducción autónoma en entornos virtuales controlados. El uso de este simulador ha permitido recrear escenarios realistas y complejos, lo que ha facilitado el entrenamiento y la validación del modelo propuesto en situaciones representativas del mundo real.

