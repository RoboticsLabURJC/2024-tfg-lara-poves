\cleardoublepage

\chapter*{Resumen\markboth{Resumen}{Resumen}}

La conducción autónoma se ha consolidado como una de las áreas más destacadas dentro de la robótica, estrechamente ligada a los avances en el desarrollo de sistemas inteligentes logrados en los últimos años. Marcados por el auge en el campo de la \ac{IA}, estos avances han revolucionado múltiples sectores gracias a su capacidad para procesar y analizar grandes volúmenes de datos en tiempo real. La \ac{IA} impulsa el desarrollo de tecnologías innovadoras basadas en la información que proporcionan sensores avanzados para navegar de forma precisa y autónoma. Estas herramientas han ampliado significativamente las capacidades de percepción de los vehículos autónomos, facilitando la toma de decisiones en situaciones dinámicas y complejas. Los modelos predictivos generados, no solo optimizan la seguridad, sino también la eficiencia energética y la sostenibilidad. Estos avances tienen diversas aplicaciones como la gestión inteligente del tráfico urbano o la automatización de flotas de transporte público.

El objetivo de este \ac{TFG} es desarrollar un sistema de conducción autónoma basado en \ac{IA} y \ac{DRL}, capaz de desenvolverse en diferentes situaciones de conducción de manera autónoma de manera segura y eficiente. Estos comportamientos incluyen el seguimiento preciso del carril a altas velocidades, la implementación de un control de crucero adaptado al tráfico, concretamente manteniendo una distancia adecuada con el vehículo que circula delante, y, finalmente, la ejecución de maniobras de adelantamiento seguras y efectivas a dicho vehículo.

Para la detección del entorno, se utilizan técnicas avanzadas de \ac{IA}, como redes neuronales de segmentación semántica y redes neuronales para la detección de las líneas de los carriles. Estas técnicas se combinan con algoritmos de \ac{DRL} para que el vehículo autónomo aprenda a tomar las mejores acciones de control en función de la información sensorial obtenida en cada instante. Este trabajo se ha llevado a cabo en el simulador CARLA, una plataforma ampliamente usada para la investigación y prueba de sistemas de conducción autónoma en entornos virtuales controlados.


