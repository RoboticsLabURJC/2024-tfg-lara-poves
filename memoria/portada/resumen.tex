\cleardoublepage

\chapter*{Resumen\markboth{Resumen}{Resumen}}

Una de las principales áreas de estudio de la robótica es la navegación autónoma. En la actualidad, ésta se lleva a cabo mediante dos tipos de 
algoritmos, enfoques clásicos que se 
basan en la planificación de rutas y trayectorias mediante algoritmos que conocen el mapa del escenario, o cada vez tomando más protagonismo aquellos basados 
en Inteligencia Artificial (IA), que consisten en aprender y adaptarse a escenarios. Dentro de la robótica aérea se están 
creando una gran número de aplicaciones variadas gracias a la capacidad de los drones, que ofrecen una perspectiva diferente del 
entorno y alcanzan zonas en los que un robot con ruedas o humanos no podrían llegar.

En este TFG se explora un método de navegación basado en IA y aprendizaje por refuerzo, 
cuyo objetivo es que el dron navegue en un área determinada, en este caso un carril de una carretera.
Este tipo de navegación, tiene muchas aplicaciones, por ejemplo, la inspección de carreteras a gran altura para analizar el estado de las mismas, tráfico, 
accidentes, primeros auxilios en desastres climatológicas.

Este trabajo propone el uso de técnicas de IA utilizando Deep Learning, como las redes neuronales de segmentación para la detección del carril y extracción 
de sus características. Así como algoritmos de aprendizaje por refuerzo, en este caso, Q-learning, para que el sistema de control del 
dron sea capaz de aprender las acciones a realizar en función de la información sensorial obtenida y sea capaz de generalizar.

Todo ello se ha realizado empleando el simulador AirSim, en el que se simula un entorno foto-realista, donde el dron realiza un seguimiento del carril. Por último, se emplea el middleware robótico ROS
para las comunicaciones entre el simulador, y el software implementado en esta solución.
